\documentclass[8pt]{extarticle}

    \usepackage[breakable]{tcolorbox}
    \usepackage{parskip} % Stop auto-indenting (to mimic markdown behaviour)

    % Basic figure setup, for now with no caption control since it's done
    % automatically by Pandoc (which extracts ![](path) syntax from Markdown).
    \usepackage{graphicx}
    % Maintain compatibility with old templates. Remove in nbconvert 6.0
    \let\Oldincludegraphics\includegraphics
    % Ensure that by default, figures have no caption (until we provide a
    % proper Figure object with a Caption API and a way to capture that
    % in the conversion process - todo).
    \usepackage{caption}
    \DeclareCaptionFormat{nocaption}{}
    \captionsetup{format=nocaption,aboveskip=0pt,belowskip=0pt}

    \usepackage{float}
    \floatplacement{figure}{H} % forces figures to be placed at the correct location
    \usepackage{xcolor} % Allow colors to be defined
    \usepackage{enumerate} % Needed for markdown enumerations to work
    \usepackage{geometry} % Used to adjust the document margins
    \usepackage{amsmath} % Equations
    \usepackage{amssymb} % Equations
    \usepackage{textcomp} % defines textquotesingle
    % Hack from http://tex.stackexchange.com/a/47451/13684:
    \AtBeginDocument{%
        \def\PYZsq{\textquotesingle}% Upright quotes in Pygmentized code
    }
    \usepackage{upquote} % Upright quotes for verbatim code
    \usepackage{eurosym} % defines \euro

    \usepackage{iftex}
    \ifPDFTeX
        \usepackage[T1]{fontenc}
        \IfFileExists{alphabeta.sty}{
              \usepackage{alphabeta}
          }{
              \usepackage[mathletters]{ucs}
              \usepackage[utf8x]{inputenc}
          }
    \else
        \usepackage{fontspec}
        \usepackage{unicode-math}
    \fi

    \usepackage{fancyvrb} % verbatim replacement that allows latex
    \usepackage{grffile} % extends the file name processing of package graphics
                         % to support a larger range
    \makeatletter % fix for old versions of grffile with XeLaTeX
    \@ifpackagelater{grffile}{2019/11/01}
    {
      % Do nothing on new versions
    }
    {
      \def\Gread@@xetex#1{%
        \IfFileExists{"\Gin@base".bb}%
        {\Gread@eps{\Gin@base.bb}}%
        {\Gread@@xetex@aux#1}%
      }
    }
    \makeatother
    \usepackage[Export]{adjustbox} % Used to constrain images to a maximum size
    \adjustboxset{max size={0.9\linewidth}{0.9\paperheight}}

    % The hyperref package gives us a pdf with properly built
    % internal navigation ('pdf bookmarks' for the table of contents,
    % internal cross-reference links, web links for URLs, etc.)
    \usepackage{hyperref}
    % The default LaTeX title has an obnoxious amount of whitespace. By default,
    % titling removes some of it. It also provides customization options.
    \usepackage{titling}
    \usepackage{longtable} % longtable support required by pandoc >1.10
    \usepackage{booktabs}  % table support for pandoc > 1.12.2
    \usepackage{array}     % table support for pandoc >= 2.11.3
    \usepackage{calc}      % table minipage width calculation for pandoc >= 2.11.1
    \usepackage[inline]{enumitem} % IRkernel/repr support (it uses the enumerate* environment)
    \usepackage[normalem]{ulem} % ulem is needed to support strikethroughs (\sout)
                                % normalem makes italics be italics, not underlines
    \usepackage{mathrsfs}



    % Colors for the hyperref package
    \definecolor{urlcolor}{rgb}{0,.145,.698}
    \definecolor{linkcolor}{rgb}{.71,0.21,0.01}
    \definecolor{citecolor}{rgb}{.12,.54,.11}

    % ANSI colors
    \definecolor{ansi-black}{HTML}{3E424D}
    \definecolor{ansi-black-intense}{HTML}{282C36}
    \definecolor{ansi-red}{HTML}{E75C58}
    \definecolor{ansi-red-intense}{HTML}{B22B31}
    \definecolor{ansi-green}{HTML}{00A250}
    \definecolor{ansi-green-intense}{HTML}{007427}
    \definecolor{ansi-yellow}{HTML}{DDB62B}
    \definecolor{ansi-yellow-intense}{HTML}{B27D12}
    \definecolor{ansi-blue}{HTML}{208FFB}
    \definecolor{ansi-blue-intense}{HTML}{0065CA}
    \definecolor{ansi-magenta}{HTML}{D160C4}
    \definecolor{ansi-magenta-intense}{HTML}{A03196}
    \definecolor{ansi-cyan}{HTML}{60C6C8}
    \definecolor{ansi-cyan-intense}{HTML}{258F8F}
    \definecolor{ansi-white}{HTML}{C5C1B4}
    \definecolor{ansi-white-intense}{HTML}{A1A6B2}
    \definecolor{ansi-default-inverse-fg}{HTML}{FFFFFF}
    \definecolor{ansi-default-inverse-bg}{HTML}{000000}

    % common color for the border for error outputs.
    \definecolor{outerrorbackground}{HTML}{FFDFDF}

    % commands and environments needed by pandoc snippets
    % extracted from the output of `pandoc -s`
    \providecommand{\tightlist}{%
      \setlength{\itemsep}{0pt}\setlength{\parskip}{0pt}}
    \DefineVerbatimEnvironment{Highlighting}{Verbatim}{commandchars=\\\{\}}
    % Add ',fontsize=\small' for more characters per line
    \newenvironment{Shaded}{}{}
    \newcommand{\KeywordTok}[1]{\textcolor[rgb]{0.00,0.44,0.13}{\textbf{{#1}}}}
    \newcommand{\DataTypeTok}[1]{\textcolor[rgb]{0.56,0.13,0.00}{{#1}}}
    \newcommand{\DecValTok}[1]{\textcolor[rgb]{0.25,0.63,0.44}{{#1}}}
    \newcommand{\BaseNTok}[1]{\textcolor[rgb]{0.25,0.63,0.44}{{#1}}}
    \newcommand{\FloatTok}[1]{\textcolor[rgb]{0.25,0.63,0.44}{{#1}}}
    \newcommand{\CharTok}[1]{\textcolor[rgb]{0.25,0.44,0.63}{{#1}}}
    \newcommand{\StringTok}[1]{\textcolor[rgb]{0.25,0.44,0.63}{{#1}}}
    \newcommand{\CommentTok}[1]{\textcolor[rgb]{0.38,0.63,0.69}{\textit{{#1}}}}
    \newcommand{\OtherTok}[1]{\textcolor[rgb]{0.00,0.44,0.13}{{#1}}}
    \newcommand{\AlertTok}[1]{\textcolor[rgb]{1.00,0.00,0.00}{\textbf{{#1}}}}
    \newcommand{\FunctionTok}[1]{\textcolor[rgb]{0.02,0.16,0.49}{{#1}}}
    \newcommand{\RegionMarkerTok}[1]{{#1}}
    \newcommand{\ErrorTok}[1]{\textcolor[rgb]{1.00,0.00,0.00}{\textbf{{#1}}}}
    \newcommand{\NormalTok}[1]{{#1}}

    % Additional commands for more recent versions of Pandoc
    \newcommand{\ConstantTok}[1]{\textcolor[rgb]{0.53,0.00,0.00}{{#1}}}
    \newcommand{\SpecialCharTok}[1]{\textcolor[rgb]{0.25,0.44,0.63}{{#1}}}
    \newcommand{\VerbatimStringTok}[1]{\textcolor[rgb]{0.25,0.44,0.63}{{#1}}}
    \newcommand{\SpecialStringTok}[1]{\textcolor[rgb]{0.73,0.40,0.53}{{#1}}}
    \newcommand{\ImportTok}[1]{{#1}}
    \newcommand{\DocumentationTok}[1]{\textcolor[rgb]{0.73,0.13,0.13}{\textit{{#1}}}}
    \newcommand{\AnnotationTok}[1]{\textcolor[rgb]{0.38,0.63,0.69}{\textbf{\textit{{#1}}}}}
    \newcommand{\CommentVarTok}[1]{\textcolor[rgb]{0.38,0.63,0.69}{\textbf{\textit{{#1}}}}}
    \newcommand{\VariableTok}[1]{\textcolor[rgb]{0.10,0.09,0.49}{{#1}}}
    \newcommand{\ControlFlowTok}[1]{\textcolor[rgb]{0.00,0.44,0.13}{\textbf{{#1}}}}
    \newcommand{\OperatorTok}[1]{\textcolor[rgb]{0.40,0.40,0.40}{{#1}}}
    \newcommand{\BuiltInTok}[1]{{#1}}
    \newcommand{\ExtensionTok}[1]{{#1}}
    \newcommand{\PreprocessorTok}[1]{\textcolor[rgb]{0.74,0.48,0.00}{{#1}}}
    \newcommand{\AttributeTok}[1]{\textcolor[rgb]{0.49,0.56,0.16}{{#1}}}
    \newcommand{\InformationTok}[1]{\textcolor[rgb]{0.38,0.63,0.69}{\textbf{\textit{{#1}}}}}
    \newcommand{\WarningTok}[1]{\textcolor[rgb]{0.38,0.63,0.69}{\textbf{\textit{{#1}}}}}


    % Define a nice break command that doesn't care if a line doesn't already
    % exist.
    \def\br{\hspace*{\fill} \\* }
    % Math Jax compatibility definitions
    \def\gt{>}
    \def\lt{<}
    \let\Oldtex\TeX
    \let\Oldlatex\LaTeX
    \renewcommand{\TeX}{\textrm{\Oldtex}}
    \renewcommand{\LaTeX}{\textrm{\Oldlatex}}
    % Document parameters
    % Document title
    \title{K\_means\_LR}





% Pygments definitions
\makeatletter
\def\PY@reset{\let\PY@it=\relax \let\PY@bf=\relax%
    \let\PY@ul=\relax \let\PY@tc=\relax%
    \let\PY@bc=\relax \let\PY@ff=\relax}
\def\PY@tok#1{\csname PY@tok@#1\endcsname}
\def\PY@toks#1+{\ifx\relax#1\empty\else%
    \PY@tok{#1}\expandafter\PY@toks\fi}
\def\PY@do#1{\PY@bc{\PY@tc{\PY@ul{%
    \PY@it{\PY@bf{\PY@ff{#1}}}}}}}
\def\PY#1#2{\PY@reset\PY@toks#1+\relax+\PY@do{#2}}

\@namedef{PY@tok@w}{\def\PY@tc##1{\textcolor[rgb]{0.73,0.73,0.73}{##1}}}
\@namedef{PY@tok@c}{\let\PY@it=\textit\def\PY@tc##1{\textcolor[rgb]{0.24,0.48,0.48}{##1}}}
\@namedef{PY@tok@cp}{\def\PY@tc##1{\textcolor[rgb]{0.61,0.40,0.00}{##1}}}
\@namedef{PY@tok@k}{\let\PY@bf=\textbf\def\PY@tc##1{\textcolor[rgb]{0.00,0.50,0.00}{##1}}}
\@namedef{PY@tok@kp}{\def\PY@tc##1{\textcolor[rgb]{0.00,0.50,0.00}{##1}}}
\@namedef{PY@tok@kt}{\def\PY@tc##1{\textcolor[rgb]{0.69,0.00,0.25}{##1}}}
\@namedef{PY@tok@o}{\def\PY@tc##1{\textcolor[rgb]{0.40,0.40,0.40}{##1}}}
\@namedef{PY@tok@ow}{\let\PY@bf=\textbf\def\PY@tc##1{\textcolor[rgb]{0.67,0.13,1.00}{##1}}}
\@namedef{PY@tok@nb}{\def\PY@tc##1{\textcolor[rgb]{0.00,0.50,0.00}{##1}}}
\@namedef{PY@tok@nf}{\def\PY@tc##1{\textcolor[rgb]{0.00,0.00,1.00}{##1}}}
\@namedef{PY@tok@nc}{\let\PY@bf=\textbf\def\PY@tc##1{\textcolor[rgb]{0.00,0.00,1.00}{##1}}}
\@namedef{PY@tok@nn}{\let\PY@bf=\textbf\def\PY@tc##1{\textcolor[rgb]{0.00,0.00,1.00}{##1}}}
\@namedef{PY@tok@ne}{\let\PY@bf=\textbf\def\PY@tc##1{\textcolor[rgb]{0.80,0.25,0.22}{##1}}}
\@namedef{PY@tok@nv}{\def\PY@tc##1{\textcolor[rgb]{0.10,0.09,0.49}{##1}}}
\@namedef{PY@tok@no}{\def\PY@tc##1{\textcolor[rgb]{0.53,0.00,0.00}{##1}}}
\@namedef{PY@tok@nl}{\def\PY@tc##1{\textcolor[rgb]{0.46,0.46,0.00}{##1}}}
\@namedef{PY@tok@ni}{\let\PY@bf=\textbf\def\PY@tc##1{\textcolor[rgb]{0.44,0.44,0.44}{##1}}}
\@namedef{PY@tok@na}{\def\PY@tc##1{\textcolor[rgb]{0.41,0.47,0.13}{##1}}}
\@namedef{PY@tok@nt}{\let\PY@bf=\textbf\def\PY@tc##1{\textcolor[rgb]{0.00,0.50,0.00}{##1}}}
\@namedef{PY@tok@nd}{\def\PY@tc##1{\textcolor[rgb]{0.67,0.13,1.00}{##1}}}
\@namedef{PY@tok@s}{\def\PY@tc##1{\textcolor[rgb]{0.73,0.13,0.13}{##1}}}
\@namedef{PY@tok@sd}{\let\PY@it=\textit\def\PY@tc##1{\textcolor[rgb]{0.73,0.13,0.13}{##1}}}
\@namedef{PY@tok@si}{\let\PY@bf=\textbf\def\PY@tc##1{\textcolor[rgb]{0.64,0.35,0.47}{##1}}}
\@namedef{PY@tok@se}{\let\PY@bf=\textbf\def\PY@tc##1{\textcolor[rgb]{0.67,0.36,0.12}{##1}}}
\@namedef{PY@tok@sr}{\def\PY@tc##1{\textcolor[rgb]{0.64,0.35,0.47}{##1}}}
\@namedef{PY@tok@ss}{\def\PY@tc##1{\textcolor[rgb]{0.10,0.09,0.49}{##1}}}
\@namedef{PY@tok@sx}{\def\PY@tc##1{\textcolor[rgb]{0.00,0.50,0.00}{##1}}}
\@namedef{PY@tok@m}{\def\PY@tc##1{\textcolor[rgb]{0.40,0.40,0.40}{##1}}}
\@namedef{PY@tok@gh}{\let\PY@bf=\textbf\def\PY@tc##1{\textcolor[rgb]{0.00,0.00,0.50}{##1}}}
\@namedef{PY@tok@gu}{\let\PY@bf=\textbf\def\PY@tc##1{\textcolor[rgb]{0.50,0.00,0.50}{##1}}}
\@namedef{PY@tok@gd}{\def\PY@tc##1{\textcolor[rgb]{0.63,0.00,0.00}{##1}}}
\@namedef{PY@tok@gi}{\def\PY@tc##1{\textcolor[rgb]{0.00,0.52,0.00}{##1}}}
\@namedef{PY@tok@gr}{\def\PY@tc##1{\textcolor[rgb]{0.89,0.00,0.00}{##1}}}
\@namedef{PY@tok@ge}{\let\PY@it=\textit}
\@namedef{PY@tok@gs}{\let\PY@bf=\textbf}
\@namedef{PY@tok@gp}{\let\PY@bf=\textbf\def\PY@tc##1{\textcolor[rgb]{0.00,0.00,0.50}{##1}}}
\@namedef{PY@tok@go}{\def\PY@tc##1{\textcolor[rgb]{0.44,0.44,0.44}{##1}}}
\@namedef{PY@tok@gt}{\def\PY@tc##1{\textcolor[rgb]{0.00,0.27,0.87}{##1}}}
\@namedef{PY@tok@err}{\def\PY@bc##1{{\setlength{\fboxsep}{\string -\fboxrule}\fcolorbox[rgb]{1.00,0.00,0.00}{1,1,1}{\strut ##1}}}}
\@namedef{PY@tok@kc}{\let\PY@bf=\textbf\def\PY@tc##1{\textcolor[rgb]{0.00,0.50,0.00}{##1}}}
\@namedef{PY@tok@kd}{\let\PY@bf=\textbf\def\PY@tc##1{\textcolor[rgb]{0.00,0.50,0.00}{##1}}}
\@namedef{PY@tok@kn}{\let\PY@bf=\textbf\def\PY@tc##1{\textcolor[rgb]{0.00,0.50,0.00}{##1}}}
\@namedef{PY@tok@kr}{\let\PY@bf=\textbf\def\PY@tc##1{\textcolor[rgb]{0.00,0.50,0.00}{##1}}}
\@namedef{PY@tok@bp}{\def\PY@tc##1{\textcolor[rgb]{0.00,0.50,0.00}{##1}}}
\@namedef{PY@tok@fm}{\def\PY@tc##1{\textcolor[rgb]{0.00,0.00,1.00}{##1}}}
\@namedef{PY@tok@vc}{\def\PY@tc##1{\textcolor[rgb]{0.10,0.09,0.49}{##1}}}
\@namedef{PY@tok@vg}{\def\PY@tc##1{\textcolor[rgb]{0.10,0.09,0.49}{##1}}}
\@namedef{PY@tok@vi}{\def\PY@tc##1{\textcolor[rgb]{0.10,0.09,0.49}{##1}}}
\@namedef{PY@tok@vm}{\def\PY@tc##1{\textcolor[rgb]{0.10,0.09,0.49}{##1}}}
\@namedef{PY@tok@sa}{\def\PY@tc##1{\textcolor[rgb]{0.73,0.13,0.13}{##1}}}
\@namedef{PY@tok@sb}{\def\PY@tc##1{\textcolor[rgb]{0.73,0.13,0.13}{##1}}}
\@namedef{PY@tok@sc}{\def\PY@tc##1{\textcolor[rgb]{0.73,0.13,0.13}{##1}}}
\@namedef{PY@tok@dl}{\def\PY@tc##1{\textcolor[rgb]{0.73,0.13,0.13}{##1}}}
\@namedef{PY@tok@s2}{\def\PY@tc##1{\textcolor[rgb]{0.73,0.13,0.13}{##1}}}
\@namedef{PY@tok@sh}{\def\PY@tc##1{\textcolor[rgb]{0.73,0.13,0.13}{##1}}}
\@namedef{PY@tok@s1}{\def\PY@tc##1{\textcolor[rgb]{0.73,0.13,0.13}{##1}}}
\@namedef{PY@tok@mb}{\def\PY@tc##1{\textcolor[rgb]{0.40,0.40,0.40}{##1}}}
\@namedef{PY@tok@mf}{\def\PY@tc##1{\textcolor[rgb]{0.40,0.40,0.40}{##1}}}
\@namedef{PY@tok@mh}{\def\PY@tc##1{\textcolor[rgb]{0.40,0.40,0.40}{##1}}}
\@namedef{PY@tok@mi}{\def\PY@tc##1{\textcolor[rgb]{0.40,0.40,0.40}{##1}}}
\@namedef{PY@tok@il}{\def\PY@tc##1{\textcolor[rgb]{0.40,0.40,0.40}{##1}}}
\@namedef{PY@tok@mo}{\def\PY@tc##1{\textcolor[rgb]{0.40,0.40,0.40}{##1}}}
\@namedef{PY@tok@ch}{\let\PY@it=\textit\def\PY@tc##1{\textcolor[rgb]{0.24,0.48,0.48}{##1}}}
\@namedef{PY@tok@cm}{\let\PY@it=\textit\def\PY@tc##1{\textcolor[rgb]{0.24,0.48,0.48}{##1}}}
\@namedef{PY@tok@cpf}{\let\PY@it=\textit\def\PY@tc##1{\textcolor[rgb]{0.24,0.48,0.48}{##1}}}
\@namedef{PY@tok@c1}{\let\PY@it=\textit\def\PY@tc##1{\textcolor[rgb]{0.24,0.48,0.48}{##1}}}
\@namedef{PY@tok@cs}{\let\PY@it=\textit\def\PY@tc##1{\textcolor[rgb]{0.24,0.48,0.48}{##1}}}

\def\PYZbs{\char`\\}
\def\PYZus{\char`\_}
\def\PYZob{\char`\{}
\def\PYZcb{\char`\}}
\def\PYZca{\char`\^}
\def\PYZam{\char`\&}
\def\PYZlt{\char`\<}
\def\PYZgt{\char`\>}
\def\PYZsh{\char`\#}
\def\PYZpc{\char`\%}
\def\PYZdl{\char`\$}
\def\PYZhy{\char`\-}
\def\PYZsq{\char`\'}
\def\PYZdq{\char`\"}
\def\PYZti{\char`\~}
% for compatibility with earlier versions
\def\PYZat{@}
\def\PYZlb{[}
\def\PYZrb{]}
\makeatother


    % For linebreaks inside Verbatim environment from package fancyvrb.
    \makeatletter
        \newbox\Wrappedcontinuationbox
        \newbox\Wrappedvisiblespacebox
        \newcommand*\Wrappedvisiblespace {\textcolor{red}{\textvisiblespace}}
        \newcommand*\Wrappedcontinuationsymbol {\textcolor{red}{\llap{\tiny$\m@th\hookrightarrow$}}}
        \newcommand*\Wrappedcontinuationindent {3ex }
        \newcommand*\Wrappedafterbreak {\kern\Wrappedcontinuationindent\copy\Wrappedcontinuationbox}
        % Take advantage of the already applied Pygments mark-up to insert
        % potential linebreaks for TeX processing.
        %        {, <, #, %, $, ' and ": go to next line.
        %        _, }, ^, &, >, - and ~: stay at end of broken line.
        % Use of \textquotesingle for straight quote.
        \newcommand*\Wrappedbreaksatspecials {%
            \def\PYGZus{\discretionary{\char`\_}{\Wrappedafterbreak}{\char`\_}}%
            \def\PYGZob{\discretionary{}{\Wrappedafterbreak\char`\{}{\char`\{}}%
            \def\PYGZcb{\discretionary{\char`\}}{\Wrappedafterbreak}{\char`\}}}%
            \def\PYGZca{\discretionary{\char`\^}{\Wrappedafterbreak}{\char`\^}}%
            \def\PYGZam{\discretionary{\char`\&}{\Wrappedafterbreak}{\char`\&}}%
            \def\PYGZlt{\discretionary{}{\Wrappedafterbreak\char`\<}{\char`\<}}%
            \def\PYGZgt{\discretionary{\char`\>}{\Wrappedafterbreak}{\char`\>}}%
            \def\PYGZsh{\discretionary{}{\Wrappedafterbreak\char`\#}{\char`\#}}%
            \def\PYGZpc{\discretionary{}{\Wrappedafterbreak\char`\%}{\char`\%}}%
            \def\PYGZdl{\discretionary{}{\Wrappedafterbreak\char`\$}{\char`\$}}%
            \def\PYGZhy{\discretionary{\char`\-}{\Wrappedafterbreak}{\char`\-}}%
            \def\PYGZsq{\discretionary{}{\Wrappedafterbreak\textquotesingle}{\textquotesingle}}%
            \def\PYGZdq{\discretionary{}{\Wrappedafterbreak\char`\"}{\char`\"}}%
            \def\PYGZti{\discretionary{\char`\~}{\Wrappedafterbreak}{\char`\~}}%
        }
        % Some characters . , ; ? ! / are not pygmentized.
        % This macro makes them "active" and they will insert potential linebreaks
        \newcommand*\Wrappedbreaksatpunct {%
            \lccode`\~`\.\lowercase{\def~}{\discretionary{\hbox{\char`\.}}{\Wrappedafterbreak}{\hbox{\char`\.}}}%
            \lccode`\~`\,\lowercase{\def~}{\discretionary{\hbox{\char`\,}}{\Wrappedafterbreak}{\hbox{\char`\,}}}%
            \lccode`\~`\;\lowercase{\def~}{\discretionary{\hbox{\char`\;}}{\Wrappedafterbreak}{\hbox{\char`\;}}}%
            \lccode`\~`\:\lowercase{\def~}{\discretionary{\hbox{\char`\:}}{\Wrappedafterbreak}{\hbox{\char`\:}}}%
            \lccode`\~`\?\lowercase{\def~}{\discretionary{\hbox{\char`\?}}{\Wrappedafterbreak}{\hbox{\char`\?}}}%
            \lccode`\~`\!\lowercase{\def~}{\discretionary{\hbox{\char`\!}}{\Wrappedafterbreak}{\hbox{\char`\!}}}%
            \lccode`\~`\/\lowercase{\def~}{\discretionary{\hbox{\char`\/}}{\Wrappedafterbreak}{\hbox{\char`\/}}}%
            \catcode`\.\active
            \catcode`\,\active
            \catcode`\;\active
            \catcode`\:\active
            \catcode`\?\active
            \catcode`\!\active
            \catcode`\/\active
            \lccode`\~`\~
        }
    \makeatother

    \let\OriginalVerbatim=\Verbatim
    \makeatletter
    \renewcommand{\Verbatim}[1][1]{%
        %\parskip\z@skip
        \sbox\Wrappedcontinuationbox {\Wrappedcontinuationsymbol}%
        \sbox\Wrappedvisiblespacebox {\FV@SetupFont\Wrappedvisiblespace}%
        \def\FancyVerbFormatLine ##1{\hsize\linewidth
            \vtop{\raggedright\hyphenpenalty\z@\exhyphenpenalty\z@
                \doublehyphendemerits\z@\finalhyphendemerits\z@
                \strut ##1\strut}%
        }%
        % If the linebreak is at a space, the latter will be displayed as visible
        % space at end of first line, and a continuation symbol starts next line.
        % Stretch/shrink are however usually zero for typewriter font.
        \def\FV@Space {%
            \nobreak\hskip\z@ plus\fontdimen3\font minus\fontdimen4\font
            \discretionary{\copy\Wrappedvisiblespacebox}{\Wrappedafterbreak}
            {\kern\fontdimen2\font}%
        }%

        % Allow breaks at special characters using \PYG... macros.
        \Wrappedbreaksatspecials
        % Breaks at punctuation characters . , ; ? ! and / need catcode=\active
        \OriginalVerbatim[#1,codes*=\Wrappedbreaksatpunct]%
    }
    \makeatother

    % Exact colors from NB
    \definecolor{incolor}{HTML}{303F9F}
    \definecolor{outcolor}{HTML}{D84315}
    \definecolor{cellborder}{HTML}{CFCFCF}
    \definecolor{cellbackground}{HTML}{F7F7F7}

    % prompt
    \makeatletter
    \newcommand{\boxspacing}{\kern\kvtcb@left@rule\kern\kvtcb@boxsep}
    \makeatother
    \newcommand{\prompt}[4]{
        {\ttfamily\llap{{\color{#2}[#3]:\hspace{3pt}#4}}\vspace{-\baselineskip}}
    }



    % Prevent overflowing lines due to hard-to-break entities
    \sloppy
    % Setup hyperref package
    \hypersetup{
      breaklinks=true,  % so long urls are correctly broken across lines
      colorlinks=true,
      urlcolor=urlcolor,
      linkcolor=linkcolor,
      citecolor=citecolor,
      }
    % Slightly bigger margins than the latex defaults

    \geometry{verbose,tmargin=0.5in,bmargin=0.5in,lmargin=0.35in,rmargin=0.35in}



\begin{document}
    \hypertarget{cscc11---introduction-to-machine-learning-fall-2022-assignment-1}{%
\section{CSCC11 - Introduction to Machine Learning, Fall 2022,
Assignment
1}\label{cscc11---introduction-to-machine-learning-fall-2022-assignment-1}}

    \hypertarget{authors}{%
\subsection{Authors}\label{authors}}

Shawn Santhoshgeorge (1006094673)\\
Anaqi Amir Razif (1005813880)

    \begin{tcolorbox}[breakable, size=fbox, boxrule=1pt, pad at break*=1mm,colback=cellbackground, colframe=cellborder]
\prompt{In}{incolor}{1}{\boxspacing}
\begin{Verbatim}[commandchars=\\\{\}]
\PY{k+kn}{import} \PY{n+nn}{pandas} \PY{k}{as} \PY{n+nn}{pd}
\PY{k+kn}{import} \PY{n+nn}{numpy} \PY{k}{as} \PY{n+nn}{np}
\PY{k+kn}{import} \PY{n+nn}{matplotlib}\PY{n+nn}{.}\PY{n+nn}{pyplot} \PY{k}{as} \PY{n+nn}{plt}
\PY{k+kn}{from} \PY{n+nn}{sklearn}\PY{n+nn}{.}\PY{n+nn}{model\PYZus{}selection} \PY{k+kn}{import} \PY{n}{train\PYZus{}test\PYZus{}split}
\PY{k+kn}{from} \PY{n+nn}{sklearn}\PY{n+nn}{.}\PY{n+nn}{cluster} \PY{k+kn}{import} \PY{n}{KMeans}
\PY{k+kn}{from} \PY{n+nn}{sklearn}\PY{n+nn}{.}\PY{n+nn}{metrics} \PY{k+kn}{import} \PY{n}{silhouette\PYZus{}score}
\PY{k+kn}{import} \PY{n+nn}{statistics}
\PY{k+kn}{from} \PY{n+nn}{scipy} \PY{k+kn}{import} \PY{n}{stats}
\end{Verbatim}
\end{tcolorbox}

    \begin{tcolorbox}[breakable, size=fbox, boxrule=1pt, pad at break*=1mm,colback=cellbackground, colframe=cellborder]
\prompt{In}{incolor}{2}{\boxspacing}
\begin{Verbatim}[commandchars=\\\{\}]
\PY{c+c1}{\PYZsh{}TO\PYZhy{}DO}
\PY{l+s+sd}{\PYZdq{}\PYZdq{}\PYZdq{}}
\PY{l+s+sd}{Read the csv file into a DataFrame \PYZhy{} df}
\PY{l+s+sd}{\PYZdq{}\PYZdq{}\PYZdq{}}
\PY{n}{df} \PY{o}{=} \PY{n}{pd}\PY{o}{.}\PY{n}{read\PYZus{}csv}\PY{p}{(}\PY{l+s+s1}{\PYZsq{}}\PY{l+s+s1}{Admission\PYZus{}Predict.csv}\PY{l+s+s1}{\PYZsq{}}\PY{p}{)}
\end{Verbatim}
\end{tcolorbox}

    \begin{tcolorbox}[breakable, size=fbox, boxrule=1pt, pad at break*=1mm,colback=cellbackground, colframe=cellborder]
\prompt{In}{incolor}{3}{\boxspacing}
\begin{Verbatim}[commandchars=\\\{\}]
\PY{l+s+sd}{\PYZdq{}\PYZdq{}\PYZdq{}}
\PY{l+s+sd}{Print the DataFrame}
\PY{l+s+sd}{\PYZdq{}\PYZdq{}\PYZdq{}}
\PY{n}{df}
\end{Verbatim}
\end{tcolorbox}

            \begin{tcolorbox}[breakable, size=fbox, boxrule=.5pt, pad at break*=1mm, opacityfill=0]
\prompt{Out}{outcolor}{3}{\boxspacing}
\begin{Verbatim}[commandchars=\\\{\}]
     Serial No.  GRE Score  TOEFL Score  University Rating  SOP  LOR   CGPA   Research   Chance of Admit
0             1        337          118                  4  4.5   4.5  9.65         1               0.92
1             2        324          107                  4  4.0   4.5  8.87         1               0.76
2             3        316          104                  3  3.0   3.5  8.00         1               0.72
3             4        322          110                  3  3.5   2.5  8.67         1               0.80
4             5        314          103                  2  2.0   3.0  8.21         0               0.65
..          {\ldots}        {\ldots}          {\ldots}                {\ldots}  {\ldots}   {\ldots}   {\ldots}       {\ldots}                {\ldots}
395         396        324          110                  3  3.5   3.5  9.04         1               0.82
396         397        325          107                  3  3.0   3.5  9.11         1               0.84
397         398        330          116                  4  5.0   4.5  9.45         1               0.91
398         399        312          103                  3  3.5   4.0  8.78         0               0.67
399         400        333          117                  4  5.0   4.0  9.66         1               0.95

[400 rows x 9 columns]
\end{Verbatim}
\end{tcolorbox}

    \begin{tcolorbox}[breakable, size=fbox, boxrule=1pt, pad at break*=1mm,colback=cellbackground, colframe=cellborder]
\prompt{In}{incolor}{4}{\boxspacing}
\begin{Verbatim}[commandchars=\\\{\}]
\PY{c+c1}{\PYZsh{}TO\PYZhy{}DO}
\PY{l+s+sd}{\PYZdq{}\PYZdq{}\PYZdq{}}
\PY{l+s+sd}{Print the length of the DataFrame.}
\PY{l+s+sd}{Print the column names of the DataFrame.}
\PY{l+s+sd}{\PYZdq{}\PYZdq{}\PYZdq{}}

\PY{n+nb}{print}\PY{p}{(}\PY{l+s+s2}{\PYZdq{}}\PY{l+s+s2}{Length of df: }\PY{l+s+s2}{\PYZdq{}}\PY{p}{,} \PY{n+nb}{len}\PY{p}{(}\PY{n}{df}\PY{p}{)}\PY{p}{)}
\PY{n+nb}{print}\PY{p}{(}\PY{l+s+s2}{\PYZdq{}}\PY{l+s+s2}{Column Names of df: }\PY{l+s+s2}{\PYZdq{}}\PY{p}{,} \PY{n+nb}{list}\PY{p}{(}\PY{n}{df}\PY{o}{.}\PY{n}{columns}\PY{p}{)}\PY{p}{)}
\end{Verbatim}
\end{tcolorbox}

    \begin{Verbatim}[commandchars=\\\{\}]
Length of df:  400
Column Names of df:  ['Serial No.', 'GRE Score', 'TOEFL Score', 'University
Rating', 'SOP', 'LOR ', 'CGPA', 'Research', 'Chance of Admit ']
    \end{Verbatim}

    \begin{tcolorbox}[breakable, size=fbox, boxrule=1pt, pad at break*=1mm,colback=cellbackground, colframe=cellborder]
\prompt{In}{incolor}{5}{\boxspacing}
\begin{Verbatim}[commandchars=\\\{\}]
\PY{c+c1}{\PYZsh{}TO\PYZhy{}DO}
\PY{l+s+sd}{\PYZdq{}\PYZdq{}\PYZdq{}}
\PY{l+s+sd}{Define an “X” array that would hold our independent features for regression purposes.}
\PY{l+s+sd}{Define a \PYZdq{}Y\PYZdq{} array that would hold our target variable.}

\PY{l+s+sd}{Print the shape of both the arrays.}
\PY{l+s+sd}{\PYZdq{}\PYZdq{}\PYZdq{}}

\PY{n}{X} \PY{o}{=} \PY{n}{df}\PY{p}{[}\PY{p}{[}\PY{l+s+s1}{\PYZsq{}}\PY{l+s+s1}{GRE Score}\PY{l+s+s1}{\PYZsq{}}\PY{p}{,} \PY{l+s+s1}{\PYZsq{}}\PY{l+s+s1}{TOEFL Score}\PY{l+s+s1}{\PYZsq{}}\PY{p}{,} \PY{l+s+s1}{\PYZsq{}}\PY{l+s+s1}{University Rating}\PY{l+s+s1}{\PYZsq{}}\PY{p}{,} \PY{l+s+s1}{\PYZsq{}}\PY{l+s+s1}{SOP}\PY{l+s+s1}{\PYZsq{}}\PY{p}{,} \PY{l+s+s1}{\PYZsq{}}\PY{l+s+s1}{LOR }\PY{l+s+s1}{\PYZsq{}}\PY{p}{,} \PY{l+s+s1}{\PYZsq{}}\PY{l+s+s1}{CGPA}\PY{l+s+s1}{\PYZsq{}}\PY{p}{,} \PY{l+s+s1}{\PYZsq{}}\PY{l+s+s1}{Research}\PY{l+s+s1}{\PYZsq{}}\PY{p}{]}\PY{p}{]}
\PY{n}{Y} \PY{o}{=} \PY{n}{df}\PY{p}{[}\PY{l+s+s1}{\PYZsq{}}\PY{l+s+s1}{Chance of Admit }\PY{l+s+s1}{\PYZsq{}}\PY{p}{]}

\PY{n+nb}{print}\PY{p}{(}\PY{l+s+s2}{\PYZdq{}}\PY{l+s+s2}{Shape of X: }\PY{l+s+s2}{\PYZdq{}}\PY{p}{,} \PY{n}{X}\PY{o}{.}\PY{n}{shape}\PY{p}{)}
\PY{n+nb}{print}\PY{p}{(}\PY{l+s+s2}{\PYZdq{}}\PY{l+s+s2}{Shape of Y: }\PY{l+s+s2}{\PYZdq{}}\PY{p}{,} \PY{n}{Y}\PY{o}{.}\PY{n}{shape}\PY{p}{)}
\end{Verbatim}
\end{tcolorbox}

    \begin{Verbatim}[commandchars=\\\{\}]
Shape of X:  (400, 7)
Shape of Y:  (400,)
    \end{Verbatim}

    \hypertarget{split-the-data}{%
\subsection{Split the data}\label{split-the-data}}

    \begin{tcolorbox}[breakable, size=fbox, boxrule=1pt, pad at break*=1mm,colback=cellbackground, colframe=cellborder]
\prompt{In}{incolor}{6}{\boxspacing}
\begin{Verbatim}[commandchars=\\\{\}]
\PY{c+c1}{\PYZsh{}TO\PYZhy{}DO}
\PY{l+s+sd}{\PYZdq{}\PYZdq{}\PYZdq{}}
\PY{l+s+sd}{Split the dataset into train dataset and test dataset.}
\PY{l+s+sd}{Set the random state to any number in order to maintain consistency while generating random numbers over several runs.}
\PY{l+s+sd}{\PYZdq{}\PYZdq{}\PYZdq{}}
\PY{n}{X\PYZus{}train}\PY{p}{,} \PY{n}{X\PYZus{}test}\PY{p}{,} \PY{n}{Y\PYZus{}train}\PY{p}{,} \PY{n}{Y\PYZus{}test} \PY{o}{=} \PY{n}{train\PYZus{}test\PYZus{}split}\PY{p}{(}\PY{n}{X}\PY{p}{,} \PY{n}{Y}\PY{p}{,} \PY{n}{train\PYZus{}size}\PY{o}{=}\PY{l+m+mf}{0.7}\PY{p}{,} \PY{n}{test\PYZus{}size}\PY{o}{=}\PY{l+m+mf}{0.3}\PY{p}{,} \PY{n}{random\PYZus{}state}\PY{o}{=}\PY{l+m+mi}{69}\PY{p}{)}
\end{Verbatim}
\end{tcolorbox}

    \hypertarget{linear-regression}{%
\subsection{Linear Regression}\label{linear-regression}}

    \begin{tcolorbox}[breakable, size=fbox, boxrule=1pt, pad at break*=1mm,colback=cellbackground, colframe=cellborder]
\prompt{In}{incolor}{7}{\boxspacing}
\begin{Verbatim}[commandchars=\\\{\}]
\PY{c+c1}{\PYZsh{}TO\PYZhy{}DO}
\PY{k}{def} \PY{n+nf}{find\PYZus{}optimal\PYZus{}parameters}\PY{p}{(}\PY{n}{x}\PY{p}{,} \PY{n}{y}\PY{p}{)}\PY{p}{:}
    \PY{l+s+sd}{\PYZdq{}\PYZdq{}\PYZdq{} Compute closed form solution for linear regression!}
\PY{l+s+sd}{    Optimal weight w* in linear regression is given by w* = (X\PYZca{}T X)\PYZca{}(\PYZhy{}1) X\PYZca{}T Y}

\PY{l+s+sd}{    Args:}
\PY{l+s+sd}{    \PYZhy{} x (ndarray (Shape: (N, D))): A NxD matrix corresponding to the inputs.}
\PY{l+s+sd}{    \PYZhy{} y (ndarray (Shape: (N, 1))): A N\PYZhy{}column vector corresponding to the outputs given the inputs.}

\PY{l+s+sd}{    Output:}
\PY{l+s+sd}{    \PYZhy{} w (ndarray (Shape: (D+1, 1))): A (D+1)x1 column vector corresponding to the bias and weights of the linear model.}
\PY{l+s+sd}{    \PYZdq{}\PYZdq{}\PYZdq{}}
    \PY{c+c1}{\PYZsh{} Pad 1\PYZsq{}s for the bias term, Why? Used for the bias}
    \PY{n}{pad\PYZus{}x} \PY{o}{=} \PY{n}{np}\PY{o}{.}\PY{n}{hstack}\PY{p}{(}\PY{p}{(}\PY{n}{np}\PY{o}{.}\PY{n}{ones}\PY{p}{(}\PY{p}{(}\PY{n}{x}\PY{o}{.}\PY{n}{shape}\PY{p}{[}\PY{l+m+mi}{0}\PY{p}{]}\PY{p}{,} \PY{l+m+mi}{1}\PY{p}{)}\PY{p}{)}\PY{p}{,} \PY{n}{x}\PY{p}{)}\PY{p}{)}

    \PY{c+c1}{\PYZsh{} Note that we could use pseudoinverse here instead: np.linalg.pinv}
    \PY{c+c1}{\PYZsh{} @ is alias for matmul}
    \PY{n}{p1} \PY{o}{=} \PY{n}{np}\PY{o}{.}\PY{n}{linalg}\PY{o}{.}\PY{n}{pinv}\PY{p}{(}\PY{n}{np}\PY{o}{.}\PY{n}{matrix}\PY{o}{.}\PY{n}{transpose}\PY{p}{(}\PY{n}{pad\PYZus{}x}\PY{p}{)} \PY{o}{@} \PY{n}{pad\PYZus{}x}\PY{p}{)} \PY{c+c1}{\PYZsh{} (X\PYZca{}T X)\PYZca{}(\PYZhy{}1)}
    \PY{n}{p2} \PY{o}{=} \PY{n}{np}\PY{o}{.}\PY{n}{matrix}\PY{o}{.}\PY{n}{transpose}\PY{p}{(}\PY{n}{pad\PYZus{}x}\PY{p}{)} \PY{o}{@} \PY{n}{y} \PY{c+c1}{\PYZsh{} X\PYZca{}T Y}
    \PY{n}{w} \PY{o}{=} \PY{n}{p1} \PY{o}{@} \PY{n}{p2}
    \PY{k}{return} \PY{n}{w}
\end{Verbatim}
\end{tcolorbox}

    \hypertarget{train-linear-regression-model-using-training-data}{%
\subsubsection{Train linear regression model using training
data}\label{train-linear-regression-model-using-training-data}}

    \begin{tcolorbox}[breakable, size=fbox, boxrule=1pt, pad at break*=1mm,colback=cellbackground, colframe=cellborder]
\prompt{In}{incolor}{8}{\boxspacing}
\begin{Verbatim}[commandchars=\\\{\}]
\PY{c+c1}{\PYZsh{}TO\PYZhy{}DO}
\PY{k}{def} \PY{n+nf}{get\PYZus{}pred\PYZus{}Y}\PY{p}{(}\PY{n}{trained\PYZus{}w}\PY{p}{,} \PY{n}{X\PYZus{}pred}\PY{p}{)}\PY{p}{:}
    \PY{l+s+sd}{\PYZdq{}\PYZdq{}\PYZdq{} Return predicted Y}
\PY{l+s+sd}{    Args:}
\PY{l+s+sd}{    \PYZhy{} trained\PYZus{}w (ndarray (Shape: (D+1, 1))): A (D+1)x1 column vector containing linear regression weights.}
\PY{l+s+sd}{    \PYZhy{} X\PYZus{}pred (ndarray (Shape: (N, D))): A NxD matrix corresponding to the prediction inputs.}

\PY{l+s+sd}{    Output:}
\PY{l+s+sd}{    \PYZhy{} pred\PYZus{}Y (ndarray (Shape: (N, 1))): A Nx1 column vector corresponding to the predicted outputs.}
\PY{l+s+sd}{    \PYZdq{}\PYZdq{}\PYZdq{}}
    \PY{c+c1}{\PYZsh{} Pad 1\PYZsq{}s for the bias term}
    \PY{n}{pad\PYZus{}x} \PY{o}{=} \PY{n}{np}\PY{o}{.}\PY{n}{hstack}\PY{p}{(}\PY{p}{(}\PY{n}{np}\PY{o}{.}\PY{n}{ones}\PY{p}{(}\PY{p}{(}\PY{n}{X\PYZus{}pred}\PY{o}{.}\PY{n}{shape}\PY{p}{[}\PY{l+m+mi}{0}\PY{p}{]}\PY{p}{,} \PY{l+m+mi}{1}\PY{p}{)}\PY{p}{)}\PY{p}{,} \PY{n}{X\PYZus{}pred}\PY{p}{)}\PY{p}{)}

    \PY{n}{pred\PYZus{}Y} \PY{o}{=} \PY{n}{pad\PYZus{}x} \PY{o}{@} \PY{n}{trained\PYZus{}w}
    \PY{k}{return} \PY{n}{pred\PYZus{}Y}
\end{Verbatim}
\end{tcolorbox}

{\newpage}

    \hypertarget{define-these-metrics-and-discuss-why-one-would-be-preferred-over-the-other}{%
\paragraph{Define these metrics and discuss why one would be preferred
over the other ? {\newline}}\label{define-these-metrics-and-discuss-why-one-would-be-preferred-over-the-other}}

\noindent

The Mean Absolute Error (MAE) is defined as the following
\(\text{MAE} = \dfrac{\sum_{i=1}^N |y_i - f(x_i)|}{N}\) and the Mean
Squared Error (MSE) is defined as the following
\(\text{MSE} = \dfrac{\sum_{i=1}^N (y_i - f(x_i))^2}{N}\). The MAE is
the average absolute error between the actual and predicted values and
the MSE is the average squared error between the actual and predicted
values. They both can be used to get an overall performance of the model
compared to the dataset but, MSE is preferred over MAE since it helps to
point out large errors to a greater extent since it squares the error
value.

    \begin{tcolorbox}[breakable, size=fbox, boxrule=1pt, pad at break*=1mm,colback=cellbackground, colframe=cellborder]
\prompt{In}{incolor}{9}{\boxspacing}
\begin{Verbatim}[commandchars=\\\{\}]
\PY{c+c1}{\PYZsh{}TO\PYZhy{}DO}
\PY{k}{def} \PY{n+nf}{get\PYZus{}mae}\PY{p}{(}\PY{n}{Y\PYZus{}truth}\PY{p}{,} \PY{n}{Y\PYZus{}pred}\PY{p}{)}\PY{p}{:}
    \PY{l+s+sd}{\PYZdq{}\PYZdq{}\PYZdq{} Return Mean absolute error}
\PY{l+s+sd}{    Args:}
\PY{l+s+sd}{    \PYZhy{} Y\PYZus{}truth (ndarray (Shape: (N, 1))): A Nx1 column vector corresponding to the actual outputs.}
\PY{l+s+sd}{    \PYZhy{} Y\PYZus{}pred (ndarray (Shape: (N, 1))): A Nx1 column vector corresponding to the predicted outputs.}

\PY{l+s+sd}{    Output:}
\PY{l+s+sd}{    \PYZhy{} MAE (ndarray (Shape: (1,))).}
\PY{l+s+sd}{    \PYZdq{}\PYZdq{}\PYZdq{}}

    \PY{l+s+s1}{\PYZsq{}}\PY{l+s+s1}{check if both inputs are of the same shape}\PY{l+s+s1}{\PYZsq{}}
    \PY{k}{assert} \PY{n}{Y\PYZus{}truth}\PY{o}{.}\PY{n}{shape} \PY{o}{==} \PY{n}{Y\PYZus{}pred}\PY{o}{.}\PY{n}{shape}\PY{p}{,} \PY{l+s+sa}{f}\PY{l+s+s2}{\PYZdq{}}\PY{l+s+s2}{Number of Actual should equal the Number of Predicted Outputs, but }\PY{l+s+si}{\PYZob{}}\PY{n}{Y\PYZus{}truth}\PY{o}{.}\PY{n}{shape}\PY{l+s+si}{\PYZcb{}}\PY{l+s+s2}{ != }\PY{l+s+si}{\PYZob{}}\PY{n}{Y\PYZus{}pred}\PY{o}{.}\PY{n}{shape}\PY{l+s+si}{\PYZcb{}}\PY{l+s+s2}{\PYZdq{}}

    \PY{k}{return} \PY{n}{np}\PY{o}{.}\PY{n}{mean}\PY{p}{(}\PY{n}{np}\PY{o}{.}\PY{n}{absolute}\PY{p}{(}\PY{n}{Y\PYZus{}truth} \PY{o}{\PYZhy{}} \PY{n}{Y\PYZus{}pred}\PY{p}{)}\PY{p}{)}

\PY{k}{def} \PY{n+nf}{get\PYZus{}mse}\PY{p}{(}\PY{n}{Y\PYZus{}truth}\PY{p}{,} \PY{n}{Y\PYZus{}pred}\PY{p}{)}\PY{p}{:}
    \PY{l+s+sd}{\PYZdq{}\PYZdq{}\PYZdq{} Return Mean squared error}
\PY{l+s+sd}{    Args:}
\PY{l+s+sd}{    \PYZhy{} Y\PYZus{}truth (ndarray (Shape: (N, 1))): A Nx1 column vector corresponding to the actual outputs.}
\PY{l+s+sd}{    \PYZhy{} Y\PYZus{}pred (ndarray (Shape: (N, 1))): A Nx1 column vector corresponding to the predicted outputs.}

\PY{l+s+sd}{    Output:}
\PY{l+s+sd}{    \PYZhy{} MSE (ndarray (Shape: (1,))).}
\PY{l+s+sd}{    \PYZdq{}\PYZdq{}\PYZdq{}}

    \PY{l+s+s1}{\PYZsq{}}\PY{l+s+s1}{check if both inputs are of the same shape}\PY{l+s+s1}{\PYZsq{}}
    \PY{k}{assert} \PY{n}{Y\PYZus{}truth}\PY{o}{.}\PY{n}{shape} \PY{o}{==} \PY{n}{Y\PYZus{}pred}\PY{o}{.}\PY{n}{shape}\PY{p}{,} \PY{l+s+sa}{f}\PY{l+s+s2}{\PYZdq{}}\PY{l+s+s2}{Number of Actual should equal the Number of Predicted Outputs, but }\PY{l+s+si}{\PYZob{}}\PY{n}{Y\PYZus{}truth}\PY{o}{.}\PY{n}{shape}\PY{l+s+si}{\PYZcb{}}\PY{l+s+s2}{ != }\PY{l+s+si}{\PYZob{}}\PY{n}{Y\PYZus{}pred}\PY{o}{.}\PY{n}{shape}\PY{l+s+si}{\PYZcb{}}\PY{l+s+s2}{\PYZdq{}}

    \PY{k}{return} \PY{n}{np}\PY{o}{.}\PY{n}{mean}\PY{p}{(}\PY{n}{np}\PY{o}{.}\PY{n}{square}\PY{p}{(}\PY{n}{Y\PYZus{}truth} \PY{o}{\PYZhy{}} \PY{n}{Y\PYZus{}pred}\PY{p}{)}\PY{p}{)}
\end{Verbatim}
\end{tcolorbox}

    \hypertarget{get-predictions-on-train-data}{%
\subsubsection{Get predictions on train
data}\label{get-predictions-on-train-data}}

    \begin{tcolorbox}[breakable, size=fbox, boxrule=1pt, pad at break*=1mm,colback=cellbackground, colframe=cellborder]
\prompt{In}{incolor}{10}{\boxspacing}
\begin{Verbatim}[commandchars=\\\{\}]
\PY{n}{w\PYZus{}optimal} \PY{o}{=} \PY{n}{find\PYZus{}optimal\PYZus{}parameters}\PY{p}{(}\PY{n}{X\PYZus{}train}\PY{p}{,} \PY{n}{Y\PYZus{}train}\PY{p}{)}
\PY{n+nb}{print}\PY{p}{(}\PY{n}{w\PYZus{}optimal}\PY{p}{)}
\end{Verbatim}
\end{tcolorbox}

    \begin{Verbatim}[commandchars=\\\{\}]
[-1.13911877  0.00129245  0.00299385  0.00304951  0.00153358  0.01990276
  0.12032817  0.03328811]
    \end{Verbatim}

    \begin{tcolorbox}[breakable, size=fbox, boxrule=1pt, pad at break*=1mm,colback=cellbackground, colframe=cellborder]
\prompt{In}{incolor}{11}{\boxspacing}
\begin{Verbatim}[commandchars=\\\{\}]
\PY{n}{pred\PYZus{}Y} \PY{o}{=} \PY{n}{get\PYZus{}pred\PYZus{}Y}\PY{p}{(}\PY{n}{w\PYZus{}optimal}\PY{p}{,} \PY{n}{X\PYZus{}train}\PY{p}{)}
\PY{n+nb}{print}\PY{p}{(}\PY{l+s+s1}{\PYZsq{}}\PY{l+s+s1}{Train Error (MSE): }\PY{l+s+s1}{\PYZsq{}}\PY{p}{,} \PY{n}{get\PYZus{}mse}\PY{p}{(}\PY{n}{Y\PYZus{}train}\PY{o}{.}\PY{n}{to\PYZus{}numpy}\PY{p}{(}\PY{p}{)}\PY{p}{,} \PY{n}{pred\PYZus{}Y}\PY{p}{)}\PY{p}{)}
\PY{n+nb}{print}\PY{p}{(}\PY{l+s+s1}{\PYZsq{}}\PY{l+s+s1}{Train Error (MAE): }\PY{l+s+s1}{\PYZsq{}}\PY{p}{,} \PY{n}{get\PYZus{}mae}\PY{p}{(}\PY{n}{Y\PYZus{}train}\PY{o}{.}\PY{n}{to\PYZus{}numpy}\PY{p}{(}\PY{p}{)}\PY{p}{,} \PY{n}{pred\PYZus{}Y}\PY{p}{)}\PY{p}{)}
\end{Verbatim}
\end{tcolorbox}

    \begin{Verbatim}[commandchars=\\\{\}]
Train Error (MSE):  0.004147808502232658
Train Error (MAE):  0.045728954899761275
    \end{Verbatim}

    \hypertarget{get-predictions-and-performance-on-test-data}{%
\subsubsection{Get predictions and performance on test
data}\label{get-predictions-and-performance-on-test-data}}

    \begin{tcolorbox}[breakable, size=fbox, boxrule=1pt, pad at break*=1mm,colback=cellbackground, colframe=cellborder]
\prompt{In}{incolor}{12}{\boxspacing}
\begin{Verbatim}[commandchars=\\\{\}]
\PY{n}{pred\PYZus{}Y} \PY{o}{=} \PY{n}{get\PYZus{}pred\PYZus{}Y}\PY{p}{(}\PY{n}{w\PYZus{}optimal}\PY{p}{,} \PY{n}{X\PYZus{}test}\PY{p}{)}
\PY{n+nb}{print}\PY{p}{(}\PY{l+s+s1}{\PYZsq{}}\PY{l+s+s1}{Test Error (MSE): }\PY{l+s+s1}{\PYZsq{}}\PY{p}{,} \PY{n}{get\PYZus{}mse}\PY{p}{(}\PY{n}{Y\PYZus{}test}\PY{o}{.}\PY{n}{to\PYZus{}numpy}\PY{p}{(}\PY{p}{)}\PY{p}{,} \PY{n}{pred\PYZus{}Y}\PY{p}{)}\PY{p}{)}
\PY{n+nb}{print}\PY{p}{(}\PY{l+s+s1}{\PYZsq{}}\PY{l+s+s1}{Test Error (MAE): }\PY{l+s+s1}{\PYZsq{}}\PY{p}{,} \PY{n}{get\PYZus{}mae}\PY{p}{(}\PY{n}{Y\PYZus{}test}\PY{o}{.}\PY{n}{to\PYZus{}numpy}\PY{p}{(}\PY{p}{)}\PY{p}{,} \PY{n}{pred\PYZus{}Y}\PY{p}{)}\PY{p}{)}
\end{Verbatim}
\end{tcolorbox}

    \begin{Verbatim}[commandchars=\\\{\}]
Test Error (MSE):  0.003748193147899049
Test Error (MAE):  0.04226151047842842
    \end{Verbatim}

{\newpage}

    \hypertarget{report-the-corresponding-mae-and-mse-values}{%
\paragraph{Report the corresponding MAE and MSE values {\newline}}\label{report-the-corresponding-mae-and-mse-values}}

\noindent

The Train Error for MAE and MSE is approximately as follows:

\begin{longtable}[]{@{}ll@{}}
\toprule
MSE & MAE\tabularnewline
\midrule
\endhead
0.004147808502232658 & 0.045728954899761275\tabularnewline
\bottomrule
\end{longtable}

The Test Error for MAE and MSE is approximately as follows:

\begin{longtable}[]{@{}ll@{}}
\toprule
MSE & MAE\tabularnewline
\midrule
\endhead
0.003748193147899049 & 0.04226151047842842\tabularnewline
\bottomrule
\end{longtable}

    \hypertarget{silouette-coefficient}{%
\subsection{Silouette Coefficient}\label{silouette-coefficient}}

    \begin{tcolorbox}[breakable, size=fbox, boxrule=1pt, pad at break*=1mm,colback=cellbackground, colframe=cellborder]
\prompt{In}{incolor}{13}{\boxspacing}
\begin{Verbatim}[commandchars=\\\{\}]
\PY{c+c1}{\PYZsh{}\PYZsh{} TO\PYZhy{}DO}
\PY{n}{n\PYZus{}silhouette} \PY{o}{=} \PY{p}{[}\PY{p}{]}

\PY{n}{kmeans\PYZus{}kwargs}\PY{o}{=} \PY{p}{\PYZob{}}
    \PY{l+s+s2}{\PYZdq{}}\PY{l+s+s2}{init}\PY{l+s+s2}{\PYZdq{}}\PY{p}{:}\PY{l+s+s2}{\PYZdq{}}\PY{l+s+s2}{k\PYZhy{}means++}\PY{l+s+s2}{\PYZdq{}}\PY{p}{,}
    \PY{l+s+s2}{\PYZdq{}}\PY{l+s+s2}{n\PYZus{}init}\PY{l+s+s2}{\PYZdq{}}\PY{p}{:}\PY{l+m+mi}{30}\PY{p}{,}
    \PY{l+s+s2}{\PYZdq{}}\PY{l+s+s2}{max\PYZus{}iter}\PY{l+s+s2}{\PYZdq{}}\PY{p}{:}\PY{l+m+mi}{250}\PY{p}{,}
    \PY{l+s+s2}{\PYZdq{}}\PY{l+s+s2}{random\PYZus{}state}\PY{l+s+s2}{\PYZdq{}}\PY{p}{:}\PY{l+m+mi}{2}
\PY{p}{\PYZcb{}}

\PY{l+s+sd}{\PYZdq{}\PYZdq{}\PYZdq{}}
\PY{l+s+sd}{Perform the following steps:}

\PY{l+s+sd}{1. Loop over the various possible K values you wish to test}
\PY{l+s+sd}{2. Initialize a K means object.}
\PY{l+s+sd}{3. Fit the training data on the K means object.}
\PY{l+s+sd}{4. Use the silhouette score method available from the sklearn metrics.}
\PY{l+s+sd}{5. Append the score to the silhouetter\PYZus{}coefficients list.}
\PY{l+s+sd}{6. Display the the silhouette coefficient associated with each value of K.}
\PY{l+s+sd}{\PYZdq{}\PYZdq{}\PYZdq{}}

\PY{k}{for} \PY{n}{k} \PY{o+ow}{in} \PY{n+nb}{range}\PY{p}{(}\PY{l+m+mi}{2}\PY{p}{,} \PY{l+m+mi}{11}\PY{p}{)}\PY{p}{:}
    \PY{n}{kmeans} \PY{o}{=} \PY{n}{KMeans}\PY{p}{(}\PY{n}{n\PYZus{}clusters}\PY{o}{=}\PY{n}{k}\PY{p}{,} \PY{o}{*}\PY{o}{*}\PY{n}{kmeans\PYZus{}kwargs}\PY{p}{)}
    \PY{n}{cluster\PYZus{}labels} \PY{o}{=} \PY{n}{kmeans}\PY{o}{.}\PY{n}{fit\PYZus{}predict}\PY{p}{(}\PY{n}{X\PYZus{}train}\PY{p}{,} \PY{n}{Y\PYZus{}train}\PY{p}{)}
    \PY{n}{silhouette\PYZus{}avg} \PY{o}{=} \PY{n}{silhouette\PYZus{}score}\PY{p}{(}\PY{n}{X\PYZus{}train}\PY{p}{,} \PY{n}{cluster\PYZus{}labels}\PY{p}{)}
    \PY{n}{n\PYZus{}silhouette}\PY{o}{.}\PY{n}{append}\PY{p}{(}\PY{n}{silhouette\PYZus{}avg}\PY{p}{)}
    \PY{n+nb}{print}\PY{p}{(}\PY{l+s+sa}{f}\PY{l+s+s2}{\PYZdq{}}\PY{l+s+s2}{For K = }\PY{l+s+si}{\PYZob{}}\PY{n}{k}\PY{l+s+si}{\PYZcb{}}\PY{l+s+s2}{. The Silhouette Score is: }\PY{l+s+si}{\PYZob{}}\PY{n}{silhouette\PYZus{}avg}\PY{l+s+si}{\PYZcb{}}\PY{l+s+s2}{\PYZdq{}}\PY{p}{)}
\end{Verbatim}
\end{tcolorbox}

    \begin{Verbatim}[commandchars=\\\{\}]
For K = 2. The Silhouette Score is: 0.523697003020886
For K = 3. The Silhouette Score is: 0.46196799620849327
For K = 4. The Silhouette Score is: 0.46609629756368104
For K = 5. The Silhouette Score is: 0.4114277091098109
For K = 6. The Silhouette Score is: 0.40338552975455844
For K = 7. The Silhouette Score is: 0.3847943002802233
For K = 8. The Silhouette Score is: 0.3439602276947788
For K = 9. The Silhouette Score is: 0.3374920312014737
For K = 10. The Silhouette Score is: 0.32371773649768576
    \end{Verbatim}

    \hypertarget{for-values-of-k-in-text2-10.-which-value-would-be-the-most-appropriate}{%
\paragraph{\texorpdfstring{For values of K \(\in [\text{2, 10}]\). Which value would be the most
appropriate ? {\newline}}{For values of K \textbackslash in {[}\textbackslash text\{2, 10\}{]}. Which value would be the most appropriate ? }}\label{for-values-of-k-in-text2-10.-which-value-would-be-the-most-appropriate}}

\noindent

From above we can see that the highest value resulting from the Silhouette coefficient analysis is approximately 0.5237 for K = 2. So the most appropriate value would be K = 2.

{\newpage}

    \hypertarget{k-means}{%
\section{K Means}\label{k-means}}

    \begin{tcolorbox}[breakable, size=fbox, boxrule=1pt, pad at break*=1mm,colback=cellbackground, colframe=cellborder]
\prompt{In}{incolor}{14}{\boxspacing}
\begin{Verbatim}[commandchars=\\\{\}]
\PY{c+c1}{\PYZsh{}TO\PYZhy{}DO}
\PY{c+c1}{\PYZsh{} Set the number of clusters based on the silhouette coefficient analysis}
\PY{n}{N\PYZus{}CLUSTERS} \PY{o}{=} \PY{l+m+mi}{2}

\PY{n}{kmeans} \PY{o}{=} \PY{n}{KMeans}\PY{p}{(}
    \PY{n}{init}\PY{o}{=}\PY{l+s+s2}{\PYZdq{}}\PY{l+s+s2}{k\PYZhy{}means++}\PY{l+s+s2}{\PYZdq{}}\PY{p}{,}
    \PY{n}{n\PYZus{}clusters}\PY{o}{=}\PY{n}{N\PYZus{}CLUSTERS} \PY{p}{,} \PY{c+c1}{\PYZsh{}Input the value you configured using the Silhouette coefficient analysis.}
    \PY{n}{n\PYZus{}init}\PY{o}{=}\PY{l+m+mi}{30}\PY{p}{,}
    \PY{n}{max\PYZus{}iter}\PY{o}{=}\PY{l+m+mi}{250}\PY{p}{,}
    \PY{n}{random\PYZus{}state}\PY{o}{=}\PY{l+m+mi}{2}
\PY{p}{)}

\PY{c+c1}{\PYZsh{}TO\PYZhy{}DO}
\PY{c+c1}{\PYZsh{} Fit to the training data}
\PY{n}{kmeans}\PY{o}{.}\PY{n}{fit}\PY{p}{(}\PY{n}{X\PYZus{}train}\PY{o}{.}\PY{n}{to\PYZus{}numpy}\PY{p}{(}\PY{p}{)}\PY{p}{,} \PY{n}{Y\PYZus{}train}\PY{o}{.}\PY{n}{to\PYZus{}numpy}\PY{p}{(}\PY{p}{)}\PY{p}{)}

\PY{c+c1}{\PYZsh{}TO\PYZhy{}DO}
\PY{c+c1}{\PYZsh{} Add the features and the training data you used to the variable below.}
\PY{n}{training\PYZus{}df\PYZus{}clustered} \PY{o}{=} \PY{n}{X\PYZus{}train}\PY{o}{.}\PY{n}{assign}\PY{p}{(}\PY{n}{cluster}\PY{o}{=}\PY{n}{kmeans}\PY{o}{.}\PY{n}{labels\PYZus{}}\PY{p}{)}

\PY{c+c1}{\PYZsh{}TO\PYZhy{}DO}
\PY{c+c1}{\PYZsh{} Predict clusters for the training data}
\PY{n}{train\PYZus{}cluster} \PY{o}{=} \PY{n}{kmeans}\PY{o}{.}\PY{n}{predict}\PY{p}{(}\PY{n}{X\PYZus{}train}\PY{o}{.}\PY{n}{to\PYZus{}numpy}\PY{p}{(}\PY{p}{)}\PY{p}{)}

\PY{c+c1}{\PYZsh{}TO\PYZhy{}DO}
\PY{c+c1}{\PYZsh{} Add the target and predicted clusters to the training DataFrame}
\PY{n}{training\PYZus{}df\PYZus{}clustered}\PY{p}{[}\PY{l+s+s1}{\PYZsq{}}\PY{l+s+s1}{cluster}\PY{l+s+s1}{\PYZsq{}}\PY{p}{]} \PY{o}{=} \PY{n}{train\PYZus{}cluster}

\PY{n}{X\PYZus{}train\PYZus{}clusters\PYZus{}df} \PY{o}{=} \PY{p}{[}\PY{p}{]}
\PY{k}{for} \PY{n}{i} \PY{o+ow}{in} \PY{n+nb}{range}\PY{p}{(}\PY{n}{N\PYZus{}CLUSTERS}\PY{p}{)}\PY{p}{:}
    \PY{n}{X\PYZus{}train\PYZus{}clusters\PYZus{}df}\PY{o}{.}\PY{n}{append}\PY{p}{(}\PY{n}{training\PYZus{}df\PYZus{}clustered}\PY{p}{[}\PY{n}{training\PYZus{}df\PYZus{}clustered}\PY{p}{[}\PY{l+s+s1}{\PYZsq{}}\PY{l+s+s1}{cluster}\PY{l+s+s1}{\PYZsq{}}\PY{p}{]}\PY{o}{==}\PY{n}{i}\PY{p}{]}\PY{p}{)}
\end{Verbatim}
\end{tcolorbox}

{\newpage}

    \hypertarget{building-linear-regression-for-our-clusters}{%
\section{Building Linear Regression for our
clusters}\label{building-linear-regression-for-our-clusters}}

    \begin{tcolorbox}[breakable, size=fbox, boxrule=1pt, pad at break*=1mm,colback=cellbackground, colframe=cellborder]
\prompt{In}{incolor}{15}{\boxspacing}
\begin{Verbatim}[commandchars=\\\{\}]
\PY{k+kn}{from} \PY{n+nn}{sklearn}\PY{n+nn}{.}\PY{n+nn}{linear\PYZus{}model} \PY{k+kn}{import} \PY{n}{LinearRegression}

\PY{l+s+sd}{\PYZdq{}\PYZdq{}\PYZdq{}}
\PY{l+s+sd}{The number of clusters would be defined by the outcome of the silhouetter coefficient}
\PY{l+s+sd}{Set up the model of Linear Regression by exploring the different parameters: https://scikit\PYZhy{}learn.org/stable/modules/generated/sklearn.linear\PYZus{}model.LinearRegression.html}
\PY{l+s+sd}{train\PYZus{}clusters\PYZus{}df is a dataframe that contains both the true cluster values and the predicted cluster values. Feel free to change the variable name to something else if you have been following a different naming convention.}
\PY{l+s+sd}{\PYZdq{}\PYZdq{}\PYZdq{}}

\PY{n}{obj\PYZus{}cluster} \PY{o}{=} \PY{p}{[}\PY{p}{]}

\PY{k}{for} \PY{n}{i} \PY{o+ow}{in} \PY{n+nb}{range}\PY{p}{(}\PY{n}{N\PYZus{}CLUSTERS}\PY{p}{)}\PY{p}{:}
    \PY{c+c1}{\PYZsh{}TO\PYZhy{}DO}
    \PY{c+c1}{\PYZsh{} Initialize a Linear Regression object.}
    \PY{n}{reg\PYZus{}model} \PY{o}{=} \PY{n}{LinearRegression}\PY{p}{(}\PY{p}{)}
    \PY{c+c1}{\PYZsh{}Get the specific X\PYZus{}train values according to their predicted clusters.}
    \PY{n}{X\PYZus{}clustered\PYZus{}data} \PY{o}{=} \PY{n}{X\PYZus{}train\PYZus{}clusters\PYZus{}df}\PY{p}{[}\PY{n}{i}\PY{p}{]}\PY{o}{.}\PY{n}{drop}\PY{p}{(}\PY{n}{columns}\PY{o}{=}\PY{p}{[}\PY{l+s+s1}{\PYZsq{}}\PY{l+s+s1}{cluster}\PY{l+s+s1}{\PYZsq{}}\PY{p}{]}\PY{p}{)}
    \PY{c+c1}{\PYZsh{}Get the specific Y\PYZus{}train values according to their predicted clusters.}
    \PY{n}{Y\PYZus{}clustered\PYZus{}data} \PY{o}{=} \PY{n}{Y\PYZus{}train}\PY{p}{[}\PY{n}{X\PYZus{}clustered\PYZus{}data}\PY{o}{.}\PY{n}{index}\PY{p}{]}
    \PY{n}{obj\PYZus{}cluster}\PY{o}{.}\PY{n}{append}\PY{p}{(}\PY{n}{reg\PYZus{}model}\PY{o}{.}\PY{n}{fit}\PY{p}{(}\PY{n}{X\PYZus{}clustered\PYZus{}data}\PY{o}{.}\PY{n}{to\PYZus{}numpy}\PY{p}{(}\PY{p}{)}\PY{p}{,} \PY{n}{Y\PYZus{}clustered\PYZus{}data}\PY{o}{.}\PY{n}{to\PYZus{}numpy}\PY{p}{(}\PY{p}{)}\PY{p}{)}\PY{p}{)}
\end{Verbatim}
\end{tcolorbox}

    \begin{tcolorbox}[breakable, size=fbox, boxrule=1pt, pad at break*=1mm,colback=cellbackground, colframe=cellborder]
\prompt{In}{incolor}{16}{\boxspacing}
\begin{Verbatim}[commandchars=\\\{\}]
\PY{k}{def} \PY{n+nf}{predict\PYZus{}value}\PY{p}{(}\PY{n}{x\PYZus{}test}\PY{p}{,} \PY{n}{kmeans}\PY{p}{,} \PY{n}{cluster\PYZus{}linear}\PY{p}{)}\PY{p}{:}
  \PY{l+s+sd}{\PYZdq{}\PYZdq{}\PYZdq{}}
\PY{l+s+sd}{  Input:}
\PY{l+s+sd}{  x\PYZus{}test is the test value that you wish to predict on.}
\PY{l+s+sd}{  kmeans is the kmeans object that you have finalized to predict on the test dataset.}
\PY{l+s+sd}{  cluster\PYZus{}linear is the list of fitted models on different clusters.}

\PY{l+s+sd}{  Return:}
\PY{l+s+sd}{  linear\PYZus{}pred \PYZhy{} linear\PYZus{}pred will be type list with prediction values}
\PY{l+s+sd}{  clusters \PYZhy{} clusters\PYZus{}pred will be the prediction of clusters using k means.}

\PY{l+s+sd}{  Follow these steps:}
\PY{l+s+sd}{  1. Predict clusters using K means object on the test data.}
\PY{l+s+sd}{  2. Predict regression values using Linear Regression list.}
\PY{l+s+sd}{  3. return both the predictions.}

\PY{l+s+sd}{  \PYZdq{}\PYZdq{}\PYZdq{}}
  \PY{n}{clusters} \PY{o}{=} \PY{p}{[}\PY{p}{]}
  \PY{n}{linear\PYZus{}pred} \PY{o}{=} \PY{p}{[}\PY{p}{]}
  \PY{k}{for} \PY{n}{index}\PY{p}{,} \PY{n}{row} \PY{o+ow}{in} \PY{n}{x\PYZus{}test}\PY{o}{.}\PY{n}{iterrows}\PY{p}{(}\PY{p}{)}\PY{p}{:}
    \PY{n}{value} \PY{o}{=} \PY{p}{[}\PY{n}{row}\PY{p}{]}
    \PY{n}{cluster\PYZus{}label} \PY{o}{=} \PY{n+nb}{int}\PY{p}{(}\PY{n}{kmeans}\PY{o}{.}\PY{n}{predict}\PY{p}{(}\PY{n}{value}\PY{p}{)}\PY{p}{)}
    \PY{n}{chance\PYZus{}to\PYZus{}admit} \PY{o}{=} \PY{n+nb}{float}\PY{p}{(}\PY{n}{cluster\PYZus{}linear}\PY{p}{[}\PY{n}{cluster\PYZus{}label}\PY{p}{]}\PY{o}{.}\PY{n}{predict}\PY{p}{(}\PY{n}{value}\PY{p}{)}\PY{p}{)}
    \PY{n}{linear\PYZus{}pred}\PY{o}{.}\PY{n}{append}\PY{p}{(}\PY{n}{chance\PYZus{}to\PYZus{}admit}\PY{p}{)}
  \PY{k}{return} \PY{n}{np}\PY{o}{.}\PY{n}{asarray}\PY{p}{(}\PY{n}{linear\PYZus{}pred}\PY{p}{,} \PY{n}{dtype}\PY{o}{=}\PY{n+nb}{float}\PY{p}{)}
\end{Verbatim}
\end{tcolorbox}

{\newpage}

    \hypertarget{final-steps}{%
\section{Final Steps}\label{final-steps}}

    \begin{tcolorbox}[breakable, size=fbox, boxrule=1pt, pad at break*=1mm,colback=cellbackground, colframe=cellborder]
\prompt{In}{incolor}{17}{\boxspacing}
\begin{Verbatim}[commandchars=\\\{\}]
\PY{c+c1}{\PYZsh{}Apply the clustering\PYZhy{}based linear regression to the test set.}
\PY{n}{Y\PYZus{}svr\PYZus{}k\PYZus{}means\PYZus{}pred} \PY{o}{=} \PY{n}{predict\PYZus{}value}\PY{p}{(}\PY{n}{X\PYZus{}test}\PY{p}{,} \PY{n}{kmeans}\PY{p}{,} \PY{n}{obj\PYZus{}cluster}\PY{p}{)}
\end{Verbatim}
\end{tcolorbox}

    \begin{tcolorbox}[breakable, size=fbox, boxrule=1pt, pad at break*=1mm,colback=cellbackground, colframe=cellborder]
\prompt{In}{incolor}{18}{\boxspacing}
\begin{Verbatim}[commandchars=\\\{\}]
\PY{n+nb}{print}\PY{p}{(}\PY{l+s+s1}{\PYZsq{}}\PY{l+s+s1}{Test Error (MSE): }\PY{l+s+s1}{\PYZsq{}}\PY{p}{,} \PY{n}{get\PYZus{}mse}\PY{p}{(}\PY{n}{Y\PYZus{}test}\PY{o}{.}\PY{n}{to\PYZus{}numpy}\PY{p}{(}\PY{p}{)}\PY{p}{,} \PY{n}{Y\PYZus{}svr\PYZus{}k\PYZus{}means\PYZus{}pred}\PY{p}{)}\PY{p}{)}
\PY{n+nb}{print}\PY{p}{(}\PY{l+s+s1}{\PYZsq{}}\PY{l+s+s1}{Test Error (MAE): }\PY{l+s+s1}{\PYZsq{}}\PY{p}{,} \PY{n}{get\PYZus{}mae}\PY{p}{(}\PY{n}{Y\PYZus{}test}\PY{o}{.}\PY{n}{to\PYZus{}numpy}\PY{p}{(}\PY{p}{)}\PY{p}{,} \PY{n}{Y\PYZus{}svr\PYZus{}k\PYZus{}means\PYZus{}pred}\PY{p}{)}\PY{p}{)}
\end{Verbatim}
\end{tcolorbox}

    \begin{Verbatim}[commandchars=\\\{\}]
Test Error (MSE):  0.003594171024942518
Test Error (MAE):  0.04166689615330025
    \end{Verbatim}

    \hypertarget{report-the-corresponding-mae-and-mse-values}{%
\paragraph{Report the corresponding MAE and MSE
values}\label{report-the-corresponding-mae-and-mse-values}}

The Test Error for MAE and MSE is as follows:

\begin{longtable}[]{@{}ll@{}}
\toprule
MSE & MAE\tabularnewline
\midrule
\endhead
0.003594171024942518 & 0.04166689615330026\tabularnewline
\bottomrule
\end{longtable}

In the previous model, we assumed that all of the students belonged to one
group, but after the Silhouette Coefficient Analysis we noticed that
there are 2 clusters. After splitting then using Kmeans++, and then
implementing Linear Regression based on the 2 clusters we see that the
MSE and MAE have both drastically improved from the previous value.

    \hypertarget{provide-a-brief-discussion-regarding-the-factors-that-might-have-contributed-to-this-result}{%
\paragraph{Provide a brief discussion regarding the factors that might
have contributed to this
result}\label{provide-a-brief-discussion-regarding-the-factors-that-might-have-contributed-to-this-result}}

Most master programs are split into 2 types, one for those interested in
Research and one for those who would like to take more advanced courses.
So it can be said that there are 2 types of applicants with different
interests. Thus we can see the 2 distinct clusters from the Silhouette
Coefficient Analysis. Comparing the Linear Regression Coefficients of
each Cluster will be very clear.

\begin{center}
\begin{tabular}{|c|c|c|c|c|c|c|c| }
  \hline
    Cluster & GRE Score & TOEFL Score & University Rating & SOP & LOR & CGPA & Research \\
  \hline
    0 & -0.0001575616593 & 0.004520914248 & 0.008651619805 & 0.014163505258 & 0.00825716335 & 0.10094909980 & 0.06810396677 \\
  \hline
    1 & 0.001835465193 & 0.003454110966 & -0.0037986638654 & -0.004936007238 & 0.027089841781 & 0.13311695689 & 0.01719038474 \\
  \hline
\end{tabular}
\end{center}

As we can see that there are many differences between the Linear
Regression Coefficients for each cluster.\\
\textbf{Cluster 0:} Favours TOFEL Score, University Rating, SOP,
Research\\
\textbf{Cluster 1:} Favours GRE Score, LOR, CGPA

Looking at what each cluster favours we can conclude that what each
cluster represents\\
\textbf{Cluster 0:} Students who have done Research with a Strong
Statment of Purpose - \textbf{Research Oriented Masters Programs}\\
\textbf{Cluster 1:} Students who have done well in Academic Courses with
Strong Letter of Recommendation - \textbf{Course Based Masters Programs}

\end{document}
